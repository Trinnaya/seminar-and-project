\documentclass[a4paper, 12pt]{report}

% Preamble ---
\usepackage[margin=1.5in]{geometry} %To adjust margins
\usepackage{xcolor} %For text in colors
\usepackage{ragged2e} 
\usepackage{url}
 %\usepackage{indentfirst}
 
%\usepackage{arrayjob}
%\newarray\bibstyles
%\readarray{bibstyles}{unsrtnat&abbrvnat&dinat&plainnat&rusnat&ksfh_nat}


%\usepackage{natbib}
\usepackage[square,numbers,sort]{natbib}
%\bibliographystyle{unsrtnat}
%\bibliographystyle{abbrvnat}
%\bibliographystyle{plainnat.bst}

\renewcommand\bibname{\centering Reference}

%To declare a document title and an author(s) 
\title{EGCI491: Assignment II}
\author{Nathan Tanaboriboon} %Replace your name here
\date{\today}

\begin{document}
	\bibliographystyle{plainnat.bst}
	\maketitle
		
	\chapter*{\centering Chapter 2 \\  Literature Review}
	
	\hspace{\parindent} 
	Provide a brief description of what this chapter covers. It is typically an outline of a comprehensive literature review for the whole project.  All related papers and previous works should be reviewed. You should summarize the main contributions, techniques used, data, key findings, and research gaps of each paper.
		
	\setcounter{chapter}{2}
	\section{Comparison of Deep Learning and the Classical Machine Learning Algorithm for the Malware Detection \cite{paper_1}}
	This paper proposed a malware-detection comparison  between using Deep Neuron Network (DNN) and using Randon Forest (RF).  
	
	Four different feature sets of Malicia data are used for performance evaluation.  
	
	True positive Rate (TPR), True negative Rate (TNR), and Positive Predictive Value (PPV), including Precision and Accuracy (Acc.) were caculated and used for performance comparison. 
	
	The experiment indicated that RF performs better that DNN.  This may be due to the combination of Auto-Endocoders used for feature extraction and DNN used for feature classification , which is too complex to predict malware using opcode frequency as a feature. 
	
	The future work is the investigaion of using other machine learning techniques such as RNN, LSTM, and ESN with more advanced feature extraction approaches.
	
	\newpage
	\section{Android Malware Detection Using Static Features and Machine Learning \cite{paper_2}}
	This paper proposed a static feature-based machine learning approach for android malware detection. 
	
	A combinatoon of various  static features such as opcode, permissions, and API calls of Android Application Pakage (APK) were used and compared with using a single type of APK. 
	
	Several machine lerning technies included Linear Classifier, oosted Trees, Gaussian Naive Bayes, Decision Tree, Random Fores (RF), and Support Vector Machine (SVM) were used for malware detection and comparison. 
	
	In this work, data set used were collected from 1) Andriod Malware Dataset (AMD), 2) Kuafu Det Dataset, and Omnidroid Dataset were used.  60\% of them were used for training, and the remaning is for performance testing in terms of True Positive (TP), False Positive (FP), True Negative (TN), and False Negative (FN). 
	
	The experiment showed that Gaussin Process, RF, and Decision Tress provided the most promissing results, respectively.
	
	The future work is to apply dynamic feature extracted from APK files to filter out clssiified malware as benign.  In addtion, more advanced machine learning methods such as Deep Neuron Network will also be used with better feature selection approaches to excluse redundant and unnecessary featurwes .
	
	
	\section{Optical Character Recognition (OCR) Using OpenCV and Python: Implementation and Performance Analysis \cite{paper_3}}
	The paper outlines fundamental steps to approach the image recognition technology which can be used to extract words or human-readable characters from photos. Such technology is expected to be implemented in the project under the function \textquotedblleft Schedule Matcher \textquotedblleft.
	
	\newpage
	\section{A Review of Strategies for Designing, Administering, and Using Student Ratings of Instruction \cite{paper_4}}
	The paper provides some level of insight to how course rating system is supposed to be like, for example, a hybrid system, implementing both the basic 0-5 score rating system and spaces for comments from students who have enrolled in the courses.
	
	
	\section{Text Mining Student Comments for Teaching Performance Evaluation using VADER and Latent Dirichlet Allocation Algorithm \cite{paper_5}}
	Backing-up the additional comment function on \textquotedblleft Rate My Course \textquotedblright function, the paper discusses how unstructured comments on courses from students can provide valuable insight to the courses that ordinary stars or scores rating cannot.
	
	
	\section{The curriculum prerequisite network: a tool for visualizing and analyzing academic curricula \cite{paper_6}}
	The paper discusses on how visualizing courses and their respective prerequisites can reveal the hidden academic structure which can be used to better analyze academic syllabus, to 
	give a better academic planning suggestions to students.
	
	
	
	\section{Preference-Based Group Scheduling \cite{paper_7}}
	The paper supports your "Master Busy Map" logic. It justifies why your system shouldn't just look for any empty slot, but should perhaps weigh slots based on how many students are free, or prioritize the "subgroup" suggestion feature when a perfect match isn't found. Such
	approach to algorithm can potentially be useful for the \textquotedblleft Super Planner
	\textquotedblleft function.
	
	
	\newpage
	\bibliography{References}
	
\end{document}