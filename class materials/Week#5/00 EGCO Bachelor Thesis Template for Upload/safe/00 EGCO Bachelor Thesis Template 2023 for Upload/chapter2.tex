\chapter{LITERATURE REVIEW}

This chapter covers reviewed academic papers and journals that are deemed useful to the development of the project, which can be categorized into 3 main themes, techniques or tools, importance, and guideline. The \textquotedblleft techniques or tools \textquotedblright papers cover the techniques, algorithms, technologies and innovations that would be used to help develop the project, for example, \cite{paper_3} outlines the OCR (optical character recognition) algorithm which can be used to \textquotedblleft extract text content from images, scanned documents, and other visual media. \textquotedblright The \textquotedblleft importance \textquotedblright papers support how each function is beneficial to the quality of life of university students, for instance,\cite{paper_4} explains how course visualization in a form of graphs and network can reveal a hidden academic structure which can provide a better insight and suggestion to the students. Lastly, the \textquotedblleft guideline \textquotedblright papers provides information on how each function should operate to achieve the project's objectives.

\setcounter{chapter}{2}


\section{Optical Character Recognition (OCR) Using OpenCV and Python: Implementation and Performance Analysis \cite{paper_3}}

The paper outlines fundamental steps to approach the image recognition technology which can be used to extract words or human-readable characters from photos. Such technology is expected to be implemented in the project under the function \textquotedblleft Schedule Matcher \textquotedblleft.

\section{A Review of Strategies for Designing, Administering, and Using Student Ratings of Instruction \cite{paper_4}}

The paper provides some level of insight to how course rating system is supposed to be like, for example, a hybrid system, implementing both the basic 0-5 score rating system and spaces for comments from students who have enrolled in the courses.


\section{Text Mining Student Comments for Teaching Performance Evaluation using VADER and Latent Dirichlet Allocation Algorithm \cite{paper_5}}

Backing-up the additional comment function on \textquotedblleft Rate My Course \textquotedblright function, the paper discusses how unstructured comments on courses from students can provide valuable insight to the courses that ordinary stars or scores rating cannot.


\section{The curriculum prerequisite network: a tool for visualizing and analyzing academic curricula \cite{paper_6}}

The paper discusses on how visualizing courses and their respective prerequisites can reveal the hidden academic structure which can be used to better analyze academic syllabus, to 
give a better academic planning suggestions to students.



\section{Preference-Based Group Scheduling \cite{paper_7}}

The paper supports the "Master Busy Map" logic. It justifies why our system shouldn't just look for any empty time slot (of the Schedule Mather function), but should perhaps weigh slots based on how many students are free, or prioritize the "subgroup" suggestion feature when a perfect match isn't found. Such approach to algorithm can potentially be useful for the \textquotedblleft Schedule Matcher
\textquotedblleft function.
